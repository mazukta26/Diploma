\documentclass{beamer}
\usepackage[utf8]{inputenc} % Кодировка utf8
\usepackage[english, russian]{babel} % Языки: русский, английский
\usepackage{textcase}
\usepackage{geometry}
\usepackage{amsmath}
\usepackage{algorithm}
\usepackage{MnSymbol}
\usepackage{listings}
\usepackage{hyperref}
\usetheme{Warsaw}
\usecolortheme{beaver}

\title[XS-схемы] % (optional, only for long titles)
{XS-схемы построения тактовых подстановок блочных криптосистем: характеристики перемешивания}
\author[Миронович С.] % (optional, for multiple authors)
{Миронович Светлана\\ 5 курс 9 группа, кафедра ММАД\\~ \\Руководитель: Агиевич Сергей Валерьевич \\ Заведующий НИЛ ПБИТ\\ кандидат физико-математических наук}
\institute[БГУ] % (optional)
{
  Белорусский государственный университет\\
  Факультет прикладной математики и информатики
}
\date[2017] % (optional)
{Минск, 2017}
\subject{Computer Science}

\begin{document}
\frame{\titlepage}

  \begin{frame}
    \frametitle{$XS$ схемы}
В $XS$-схемах возможны только 2 операции:
\begin{itemize}
\item Операция исключающее ИЛИ
\item Подстановка в S-блок
\end{itemize}
$XS_1$-схемы - это схемы, где используется только один $S$-блок.
Все $XS_1$-схемы можно задать тройкой $(a, B, c)$ или расширенной матрицей:
$$
\begin{pmatrix}
b_{11} & b_{12} & \ldots & b_{1n} & a_1\\
b_{21} & b_{22} & \ldots & b_{2n} & a_2\\
\dotfill\\
b_{n1} & b_{n2} & \ldots & b_{nn} & a_n\\
c_1    & c_2    & \ldots & c_n    & 0\\
\end{pmatrix}.
$$
Тогда преобразование выглядит следующим образом:
$$y = xB + S(xa)c, x = (X_1, X_2, X_3, ..., X_n)$$.
  \end{frame}
 \begin{frame}
    \frametitle{$XS$ схемы}
Например, рассмотрим тактовую подстановку Skipjack A. Она задается следующим преобразованием:
$$X_1, X_2, X_3, X_4 \rightarrow X_4+S(X_1), S(X_1), X_2, X_3$$
А тогда ее расширенная матрица выглядит следующим образом:
$$
\begin{pmatrix}
0 & 0 & 0 & 0 & 1\\
0 & 0 & 1 & 0 & 0\\
0 & 0 & 0 & 1 & 0\\
1 & 0 & 0 & 0 & 0\\
1    & 1    & 0 & 0    & 0\\
\end{pmatrix}.
$$
  \end{frame}
  \begin{frame}
    \frametitle{Граф разностных переходов}
Пусть у нас есть $X=X_1||X_2||...||X_n$ и $X'=X_1'||X_2'||X_3'||...||X_n'$, которые после одного такта переходят в $Y=Y_1||Y_2||Y_3||...||Y_n$ и $Y'=Y_1'||Y_2'||Y_3'||...||Y_n'$. Вектора $u$ и $v$ будем называть разностными индикаторами для $X, X'$ и $Y, Y'$ соответственно, если:
$$
u_i=I\left\{X_i\ne X'_i\right\}=\left\{ \begin{array}{c}
1,X_i\ne X'_i \\
0,X_i=X'_i \end{array}
\right.i=1,\dots , n.
$$
$$
v_i=I\left\{Y_i\ne Y'_i\right\}=\left\{ \begin{array}{c}
1,Y_i\ne Y'_i \\
0,Y_i=Y'_i \end{array}
\right.i=1,\dots , n.
$$
Тогда говорят, что из $u$ существует разностный переход в $v$.
Также будем говорить, что $S$-блок разностно активен, если от $X$ и $X'$ в $S$-блок поступают разные вектора (или, что то же самое, когда $(a,u) > 0$).
Тогда графом разностных переходов называется граф, вершины которого - это вектора $\{0, 1\}^n$, а ребро из $u$ в $v$ существует тогда и только тогда, когда существует разностный переход из $u$ в $v$. Причем вес ребра будет 1, если $S$-блок разностно активен; иначе 0.

  \end{frame}
\begin{frame}
    \frametitle{Граф линейных переходов}
На вход тактовой подстановке поступает $X = X_1||X_2||...||X_n$, на выходе получаем $Y=E(X) = Y_1||Y_2||...||Y_n$.
Индикатором $q = (q_1, q_2, ..., q_n)$ для $a=(a_1, a_2, ..., a_n)$ назовем:
$$q_i=I\left\{a_i\ne 0\right\}=\left\{ \begin{array}{c}
1,a_i \neq 0 \\
0,a_i = 0\end{array}
\right.i=1,\dots , n.$$
Если следующее соотношение выполняется для всех $X_i, i=\overline{1,n}$:
$$\alpha_1X_1 + \alpha_2X_2 + ... + \alpha_nX_n = \beta_1Y_1 + \beta_2Y_2 + ... + \beta_nY_n$$
Или, записывая иначе:
$$\gamma_1X_1 + \gamma_2X_2 + ... + \gamma_nX_n = \gamma S(a_1X_1 + a_2X_2 + ... + a_nX_n)$$
причем $q$ идентификатор для $\alpha$, $r$ идентификатор для $\beta$, тогда говорят, что из $q$ совершен линейный переход в $r$. И тогда $S$-блок будет называться линейно активным, если $\gamma \neq 0$.
  \end{frame}
\begin{frame}
    \frametitle{Граф линейных переходов}
Графом линейных переходов будем называть такой ориентированный весовой граф, у которого вершинами являются вектора $\{0, 1\}^n$, и из вершины $q$ есть ребро в вершину $r$ тогда и только тогда, когда существуют такие $\alpha = (\alpha_1, \alpha_2, ..., \alpha_n)$, $\beta = (\beta_1, \beta_2, ..., \beta_n)$, что для любых $X_1, X_2, ..., X_n$ выполняется $\alpha_1X_1 + \alpha_2X_2 + ... + \alpha_nX_n = \beta_1Y_1 + \beta_2Y_2 + ... + \beta_nY_n$, и притом $q, r$ являются соответственно индикаторами для $\alpha, \beta$. Ребро из $q$ в $r$ имеет вес 1, если $S$-блок является в данном соотношении линейно активным; иначе вес ребра равен 0.
  \end{frame}
\begin{frame}
    \frametitle{Легчайший путь длины $k$ в графе}
\begin{algorithm}[H]
\caption{Модифицированный Алгоритм Беллмана-Форда}
\label{diff_graph_construct}
\textbf{Вход:} Граф $G$.\\
\textbf{Выход:} Массив $\pi_l$, в котором $\pi_l[v]$ содержит самый легкий путь длины $l$ до вершины $v$.\\
Шаг 0. Создаем массив $\pi_0$ длины $n$ и заполняем его $0$. \\
Шаг 1. Для $k$ от 1 до $l$ делаем следующее:\\
Шаг 1.1 Создаем массив $\pi_k$ и заполняем его $+\infty$,\\
Шаг 1.2 Для всех ребер $(u, v) \in E:$\\
Шаг 2.1.1 Если $\pi_k[v] > \pi_{k-1}[u] + w(u,v)$, то $\pi_k[v] = \pi_{k-1}[u] + w(u,v)$\\
Шаг 3. Возвращаем $\pi_l$.\\
\end{algorithm}
  \end{frame}
\begin{frame}
    \frametitle{Цикл минимального среднего веса}
\begin{algorithm}[H]
\caption{Алгоритм нахождения $\Lambda$}
\label{diff_graph_construct}
\textbf{Вход:} Граф $G$\\
\textbf{Выход:} $\Lambda$.\\
Шаг 0. Вычислим все вектора $\Pi_i$ и $prev_i$ для $i = \overline{1, ITER}$ (здесь $prev_i[v]$ хранит значение, из какой вершины был совершен последний переход в вершину $v$ по легчайшему пути длины $i$).\\
Шаг 1. Присваиваем $minMean = \infty$.\\
Шаг 2.  $\forall ~v \in V, \Pi_{ITER}[v] \ne \infty$, делаем следующее:\\
Шаг 2.1. Восстанавливаем последний цикл, который содержится в легчайшем пути длины $ITER$, ведущим в $v$. \\
Шаг 2.2. Вычисляем его средний вес и записываем в $currentMean$.\\
Шаг 2.3. Если $minMean > currentMean$ , то записываем в $minMean$ значение $currentMean$.\\
Шаг 3. Возвращаем $minMean$.\\ 
\end{algorithm}
  \end{frame}
\begin{frame}
    \frametitle{Проделанная работа}
\begin{itemize}
\item Релизован класс для работы с XS1 схемами
\item Написан конвертер схемы в граф разностных переходов
\item Написан конвертер схемы в граф линейных переходов
\item Разработан и реализован алгоритм вычисления легчайшего пути заданной длины на графе
\item Разработан и реализован алгоритм нахождения цикла минимального среднего веса на графе
\end{itemize}
  \end{frame}
% etc
\end{document}