\documentclass[a4paper,14pt]{extarticle}
\usepackage[utf8]{inputenc} % Кодировка utf8
\usepackage[english, russian]{babel} % Языки: русский, английский
\usepackage{textcase}
\usepackage{geometry}
\usepackage{amsmath}
\usepackage{MnSymbol}
\usepackage{listings}
\geometry{a4paper,top=2cm,bottom=2cm,left=3cm,right=1cm}

\begin{document}
\thispagestyle{empty}
\vspace*{2cm}

\begin{center}
\MakeTextUppercase{Министерство образования Республики Беларусь}\par
\MakeTextUppercase{Белорусский Государственный Университет}\par 
\MakeTextUppercase{Факультет Прикладной Математики и Информатики}\par
Кафедра математического моделирования и анализа данных
\par
\end{center}

\vspace*{5cm}

\begin{center}
{\bf \large AS-схемы построения тактовых подстановок блочных криптосистем: характеристики перемешивания}
\end{center}

\vspace*{2cm}

\begin{center}
Дипломная работа
\end{center}

\vspace{2cm}

\begin{flushright}
\begin{tabular}{l}
Миронович Светланы Юрьевны\\
студентки 5 курса,\\
специальность <<Компьютерная безопасность>>\\
\\
Научный руководитель:\\
Заведующий НИЛ проблем 
\\безопасности информационных технологий\\
Агиевич Сергей Валерьевич
\end{tabular}
\end{flushright}

\begin{flushleft}
\begin{tabular}{r}
"\_\_" \_\_\_\_\_\_\_\_\_ 2017
\end{tabular}
\end{flushleft}

\par\vspace*{\fill}

\begin{center}
{Минск, 2017}
\end{center}
\newpage
\tableofcontents
 
\clearpage

\newpage
\section*{Введение}
\addcontentsline{toc}{section}{Введение}
\vspace*{1cm}

Современный мир невозможно представить без криптографии [1]. Телефонные звонки, сообщения в социальных сетях, интернет-банкинг - все это требует специальные средства для защиты данных каждого пользователя. Как никогда остро стоит вопрос целостности (т.е. уверенность, что данные не были изменены) и конфиденциальности (т.е. необходимость предотвращения утечки или разглашения) информации [2]. Таким образом, знание о том, как сравнивать криптосистемы и находить наилучшую криптосистему из множества является прикладным.

С другой стороны, важной проблемой является быстродействие. Некоторые приложения в интернете, такие как видеозвоки, должны не только защищать целостность и конфиденциальность, но еще и работать "на лету". В данном случае важным параметром становится количество операций, необходимое для одного такта зашифрования.

Таким образом, для ряда задач требуются криптосистемы, которые для своего количества операций являются наиболее криптостойкими по некоторой заданной метрике.

На текущий момент в криптографии есть два вида криптосистем: симметричные, которые в преобразованиях зашифрования и расшифрования используют один и тот же ключ, и асимметричные, которым для зашифрования и расшифрования требуются разные ключи. Симметричные криптосистемы в свою очередь разделяются на блочные (когда исходный текст разбивается на блоки одинаковой длины и преобразование зашифрования применяется к каждому блоку поотдельности) и поточные (когда генерируется гамма-последовательность и шифртекст получается как результат применения операции XOR над исходным текстом и гамма-последовательностью). 

Криптографическое преобразование можно задать схемой~---
ориентированным графом, который описывает ход вычислений. Дальше мы будем говорить о криптосистемах как о схемах и рассматривать свойства именно схем.

Криптосистемы можно классифицировать по операциям, которые применяются в тактовых подстановках на каждом фрагменте. Используют следующие операции:

\begin{itemize}
\item R
\newline Операция циклического сдвига
\item L
\newline Логические операции И и ИЛИ
\item S
\newline Замена на S-блоке
\item X
\newline Применение операции XOR над двумя фрагментами
\item A
\newline Сложение фрагментов длины m как чисел по модулю $2^m$
\end{itemize}

Различные сочетания этих операций дают различные типы схем. Так, криптосистемы SM4 и Skipjack принадлежат к классу XS, AES можно отнести к классу XLS, а схемы криптосистем Salsa20, ChaCha и CAST-128 относятся к классу ARX.

Целью данной работы является исследование так называемых XS схем с целью нахождения оптимальной с точки зрения как криптостойкости, так и быстродействия схемы.

\newpage
\section*{1 Обзор литературы}
\addcontentsline{toc}{section}{1 Обзор литературы}
\vspace*{1cm}


\newpage
\section*{2 Построение схем $XS$}
\addcontentsline{toc}{section}{2 Построение схем $XS$}
\vspace*{1cm}

Пусть на вход поступает двоичное слово $X$, из которого в ходе зашифрования получают $Y$. Тактовые подстановки функционируют следующим образом: $X$ разбивается на $n$ равных по длине частей $X_1, X_2, X_3, ..., X_n \in \{0, 1\}^m, X = X_1||X_2||X_3||...||X_n$. Фрагменты $X_i$ интерпретируются как вектора длины $m$ над полем $\{0, 1\}$. Затем вычисляется $Y_i, i = \overline{1,n}$ как комбинация фрагментов $X_j$ с некоторыми операциями, такими как битовый сдвиг, исключающее ИЛИ, сумма по модулю $2^m$, замена на S блоке, логическое И, логическо ИЛИ и т.д. Затем фрагменты $Y_i$ объединяются, чтобы получить выходное значение $Y: Y = Y_1||Y_2||Y_3||...||Y_n$. Здесь $n$ называют размерностью схемы, количество битов в $X_i, Y_i$ $m$ - размером фрагмента.

Рассмотрим схемы, в которых есть только операции замены на S-блоке и исключающее ИЛИ. Обозначим такие схемы как $XS$, подразумевая, что $X$ в названии $XS$ означает исключающее ИЛИ, $S$ означает замену на $S$ блоке. 

Все $XS$ схемы можно разделить на классы в зависимости от того, сколько различных $S$ блоков используется внутри схемы. Так, $XS_k$ обозначает такую $XS$-схему, в которой присутствует $k$ различных $S$ блоков. Таким образом, будет справедливо следующее: $XS = XS_1 \cup XS_2 \cup ... \cup XS_k \cup...$. Можно также провести аналогичное разделение по количеству операций исключающего ИЛИ, необходимых для одного такта, т.е. $X_lS_k$ - схема с $l$ сложениями по модулю 2 и $k$ заменами на $S$ блоке. Понятно, что значения $l, k$ характеризуют сложность схемы: чем они больше, тем схема сложнее.

Основным объектом изучения будут схемы с одним блоком $S$ и неограниченным количеством операций XOR, т.е. схемы $XS_1$. К схемам $XS_1$ можно отнести ряд тактовых подстановок, таких как Skipjack A, Skipjack B, SM4. 

Схемы $XS$ размерности $n$ можно задать в виде матрицы $B = (b_{ij})$ размерности $n \times n$, вектора-строки $c = (c_i)$ размерности $n$ и вектора-столбца $a=(a_i)$ размерности $n$ следующим образом:

$$u = a_1X_1 + a_2X_2 + ... + a_nX_n$$
$$v = S(u)$$
$$Y_i = b_{1i}X_1 + b_{2i}X_2 + ... + b_{ni}X_n + c_iv, \forall i = \overline{1,n}$$

Здесь "+" обозначает операцию сложения по модулю 2, а $b_{ij}, a_i, c_i \in \{0, 1\}$, и соответственно, если $b_{ij} = 0 \vee c_j=0 \vee a_j = 0$, мы полагаем, что соответствющий вектор $X_j$ не участвует в сложении по модулю 2. В случае же, когда $b_{ij} = 1 \vee c_j=1 \vee a_j = 1$, соответствующий вектор $X_j$ включается в сложение.

Соответственно, преобразование можно задать в матричной форме:

$$y = xB + S(xa)c, x = (X_1, X_2, X_3, ..., X_n)$$.

В таком случае, видно, что вся схема $XS_1$ однозначно задается тройкой $(a, B, c)$. 

Также удобно записывать $(a, B, c)$ в виде следующей матрицы, которую будем называть расширенной:

$$
\begin{pmatrix}
b_{11} & b_{12} & \ldots & b_{1n} & a_1\\
b_{21} & b_{22} & \ldots & b_{2n} & a_2\\
\dotfill\\
b_{n1} & b_{n2} & \ldots & b_{nn} & a_n\\
c_1    & c_2    & \ldots & c_n    & 0\\
\end{pmatrix}.
$$

Рассмотрим на примере тактовой подстановки Skipjack A:

$$u = X_1$$
$$v = S(u)$$
$$Y_1 = X_4 + v$$
$$Y_2 = v$$
$$Y_3 = X_2$$
$$Y_4 = X_3$$

Или в виде расширенной матрицы:

$$
\begin{pmatrix}
0 & 0 & 0 & 0 & 1\\
0 & 0 & 1 & 0 & 0\\
0 & 0 & 0 & 1 & 0\\
1 & 0 & 0 & 0 & 0\\
1    & 1    & 0 & 0    & 0\\
\end{pmatrix}.
$$

В ходе преддипломной практики был разработан и реализован программно формат представления схем класса $XS_1$. Код, выполняющий чтение из файла матричного представления схемы и создающий объект схемы $XS_1$ можно найти в приложении. 

\newpage
\section*{3 Граф разностных переходов}
\addcontentsline{toc}{section}{3 Граф разностных переходов}
\vspace*{1cm}

Пусть у нас есть схема $XS_1$ с параметрами $(a, B, c)$. Обозначим ее как $E$,  $E(X)$ - результат подстановки $X$ в схему $E$. 

Пусть также на вход схемы поступает два различных входа $X=X_1||X_2||X_3||...||X_n$ и $X'=X_1'||X_2'||X_3'||...||X_n'$, которые соответственно после проведения одного такта схемы преобразуются в $Y=Y_1||Y_2||Y_3||...||Y_n$ и $Y'=Y_1'||Y_2'||Y_3'||...||Y_n'$. Введем два вектора, $u$ и $v$ такие, что:

$$
u_i=I\left\{X_i\ne X'_i\right\}=\left\{ \begin{array}{c}
1,X_i\ne X'_i \\
0,X_i=X'_i \end{array}
\right.i=1,\dots , n.
$$
$$
v_i=I\left\{Y_i\ne Y'_i\right\}=\left\{ \begin{array}{c}
1,Y_i\ne Y'_i \\
0,Y_i=Y'_i \end{array}
\right.i=1,\dots , n.
$$

Будем называть $u$ разностным индикатором для $X$ и $X'$, и соответственно $v$ - это разностный индикатор $Y$ и $Y'$.

Говорят, из $u$ совершен разностный переход в $v$, когда существует два таких вектора $X$ и $X'$ такие, что $u$ - это их разностный индикатор, а $v$ является разностным индикаторов векторов $E(X) = Y, E(X') = Y'$.

$S$-блок при разностном переходе называется активным, если векторное произведение $(a, u) \neq 0$ (или, что то же самое, если $Xa \neq X'a$), поскольку в таком случае на вход $S$-блоку поступали разные вектора.

Тогда графом разностных переходов будем называть следующий ориентированный весовой граф. Его вершинами являются всевозможные битовые вектора длины $n$, и из вершины $u$ проведено ребро в вершину $v$ тогда и только тогда, когда из $u$ можно совершить разностный переход в $v$. Вес у этого ребра будет 1, если $S$-блок при таком разностном переходе является активным; в противном случае вес ребра будет 0.

\end{document}
