\newpage
\section*{3 Граф разностных переходов}
\addcontentsline{toc}{section}{3 Граф разностных переходов}
\vspace*{1cm}

Пусть у нас есть схема $XS_1$ с параметрами $(a, B, c)$. Обозначим ее как $E$,  $E(X)$ - результат подстановки $X$ в схему $E$. 

Пусть также на вход схемы поступает два различных входа $X=X_1||X_2||X_3||...||X_n$ и $X'=X_1'||X_2'||X_3'||...||X_n'$, которые соответственно после проведения одного такта схемы преобразуются в $Y=Y_1||Y_2||Y_3||...||Y_n$ и $Y'=Y_1'||Y_2'||Y_3'||...||Y_n'$. Введем два вектора, $u$ и $v$ такие, что:

$$
u_i=I\left\{X_i\ne X'_i\right\}=\left\{ \begin{array}{c}
1,X_i\ne X'_i \\
0,X_i=X'_i \end{array}
\right.i=1,\dots , n.
$$
$$
v_i=I\left\{Y_i\ne Y'_i\right\}=\left\{ \begin{array}{c}
1,Y_i\ne Y'_i \\
0,Y_i=Y'_i \end{array}
\right.i=1,\dots , n.
$$

Будем называть $u$ разностным индикатором для $X$ и $X'$, и соответственно $v$ - это разностный индикатор $Y$ и $Y'$.

Говорят, из $u$ совершен разностный переход в $v$, когда существует два таких вектора $X$ и $X'$ такие, что $u$ - это их разностный индикатор, а $v$ является разностным индикаторов векторов $E(X) = Y, E(X') = Y'$.

$S$-блок при разностном переходе называется активным, если векторное произведение $(a, u) \neq 0$ (или, что то же самое, если $Xa \neq X'a$), поскольку в таком случае на вход $S$-блоку поступали разные вектора.

Тогда графом разностных переходов будем называть следующий ориентированный весовой граф. Его вершинами являются всевозможные битовые вектора длины $n$, и из вершины $u$ проведено ребро в вершину $v$ тогда и только тогда, когда из $u$ можно совершить разностный переход в $v$. Вес у этого ребра будет 1, если $S$-блок при таком разностном переходе является активным; в противном случае вес ребра будет 0.