\newpage
\section*{2 Построение схем $XS$}
\addcontentsline{toc}{section}{2 Построение схем $XS$}
\vspace*{1cm}


Пусть на вход поступает двоичное слово $X$, из которого в ходе зашифрования получают $Y$. Тактовые подстановки функционируют следующим образом: $X$ разбивается на $n$ равных по длине частей $X_1, X_2, X_3, ..., X_n \in \{0, 1\}^m, X = X_1||X_2||X_3||...||X_n$. Фрагменты $X_i$ интерпретируются как вектора длины $m$ над полем $\{0, 1\}$. Затем вычисляется $Y_i, i = \overline{1,n}$ как комбинация фрагментов $X_j$ с некоторыми операциями, такими как битовый сдвиг, исключающее ИЛИ, сумма по модулю $2^m$, замена на S блоке, логическое И, логическо ИЛИ и т.д. Затем фрагменты $Y_i$ объединяются, чтобы получить выходное значение $Y: Y = Y_1||Y_2||Y_3||...||Y_n$. Здесь $n$ называют размерностью схемы, количество битов в $X_i, Y_i$ $m$ - размером фрагмента.

Рассмотрим схемы, в которых есть только операции замены на S-блоке и исключающее ИЛИ. Обозначим такие схемы как $XS$, подразумевая, что $X$ в названии $XS$ означает исключающее ИЛИ, $S$ означает замену на $S$ блоке. 

Все $XS$ схемы можно разделить на классы в зависимости от того, сколько различных $S$ блоков используется внутри схемы. Так, $XS_k$ обозначает такую $XS$-схему, в которой присутствует $k$ различных $S$ блоков. Таким образом, будет справедливо следующее: $XS = XS_1 \cup XS_2 \cup ... \cup XS_k \cup...$. Можно также провести аналогичное разделение по количеству операций исключающего ИЛИ, необходимых для одного такта, т.е. $X_lS_k$ - схема с $l$ сложениями по модулю 2 и $k$ заменами на $S$ блоке. Понятно, что значения $l, k$ характеризуют сложность схемы: чем они больше, тем схема сложнее.

Основным объектом изучения будут схемы с одним блоком $S$ и неограниченным количеством операций XOR, т.е. схемы $XS_1$. К схемам $XS_1$ можно отнести ряд тактовых подстановок, таких как Skipjack A, Skipjack B, SM4. 

Схемы $XS$ размерности $n$ можно задать в виде матрицы $B = (b_{ij})$ размерности $n \times n$, вектора-строки $c = (c_i)$ размерности $n$ и вектора-столбца $a=(a_i)$ размерности $n$ следующим образом:

$$u = a_1X_1 + a_2X_2 + ... + a_nX_n$$
$$v = S(u)$$
$$Y_i = b_{1i}X_1 + b_{2i}X_2 + ... + b_{ni}X_n + c_iv, \forall i = \overline{1,n}$$

Здесь "+" обозначает операцию сложения по модулю 2, а $b_{ij}, a_i, c_i \in \{0, 1\}$, и соответственно, если $b_{ij} = 0 \vee c_j=0 \vee a_j = 0$, мы полагаем, что соответствющий вектор $X_j$ не участвует в сложении по модулю 2. В случае же, когда $b_{ij} = 1 \vee c_j=1 \vee a_j = 1$, соответствующий вектор $X_j$ включается в сложение.

Соответственно, преобразование можно задать в матричной форме:

$$y = xB + S(xa)c, x = (X_1, X_2, X_3, ..., X_n)$$.

В таком случае, видно, что вся схема $XS_1$ однозначно задается тройкой $(a, B, c)$. 

Также удобно записывать $(a, B, c)$ в виде следующей матрицы, которую будем называть расширенной:

$$
\begin{pmatrix}
b_{11} & b_{12} & \ldots & b_{1n} & a_1\\
b_{21} & b_{22} & \ldots & b_{2n} & a_2\\
\dotfill\\
b_{n1} & b_{n2} & \ldots & b_{nn} & a_n\\
c_1    & c_2    & \ldots & c_n    & 0\\
\end{pmatrix}.
$$

Рассмотрим на примере тактовой подстановки Skipjack A:

$$u = X_1$$
$$v = S(u)$$
$$Y_1 = X_4 + v$$
$$Y_2 = v$$
$$Y_3 = X_2$$
$$Y_4 = X_3$$

Или в виде расширенной матрицы:

$$
\begin{pmatrix}
0 & 0 & 0 & 0 & 1\\
0 & 0 & 1 & 0 & 0\\
0 & 0 & 0 & 1 & 0\\
1 & 0 & 0 & 0 & 0\\
1    & 1    & 0 & 0    & 0\\
\end{pmatrix}.
$$

В ходе преддипломной практики был разработан и реализован программно формат представления схем класса $XS_1$. Код, выполняющий чтение из файла матричного представления схемы и создающий объект схемы $XS_1$ можно найти в приложении. 
