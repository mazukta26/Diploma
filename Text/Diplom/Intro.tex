\newpage
\section*{Введение}
\addcontentsline{toc}{section}{Введение}
\vspace*{1cm}

Современный мир невозможно представить без криптографии [1]. Телефонные звонки, сообщения в социальных сетях, интернет-банкинг - все это требует специальные средства для защиты данных каждого пользователя. Как никогда остро стоит вопрос целостности (т.е. уверенность, что данные не были изменены) и конфиденциальности (т.е. необходимость предотвращения утечки или разглашения) информации [2]. Таким образом, знание о том, как сравнивать криптосистемы и находить наилучшую криптосистему из множества является прикладным.

С другой стороны, важной проблемой является быстродействие. Некоторые приложения в интернете, такие как видеозвоки, должны не только защищать целостность и конфиденциальность, но еще и работать "на лету". В данном случае важным параметром становится количество операций, необходимое для одного такта зашифрования.

Таким образом, для ряда задач требуются криптосистемы, которые для своего количества операций являются наиболее криптостойкими по некоторой заданной метрике.

На текущий момент в криптографии есть два вида криптосистем: симметричные, которые в преобразованиях зашифрования и расшифрования используют один и тот же ключ, и асимметричные, которым для зашифрования и расшифрования требуются разные ключи. Симметричные криптосистемы в свою очередь разделяются на блочные (когда исходный текст разбивается на блоки одинаковой длины и преобразование зашифрования применяется к каждому блоку поотдельности) и поточные (когда генерируется гамма-последовательность и шифртекст получается как результат применения операции XOR над исходным текстом и гамма-последовательностью). 

Криптографическое преобразование можно задать схемой~---
ориентированным графом, который описывает ход вычислений. Дальше мы будем говорить о криптосистемах как о схемах и рассматривать свойства именно схем.

Криптосистемы можно классифицировать по операциям, которые применяются в тактовых подстановках на каждом фрагменте. Используют следующие операции:

\begin{itemize}
\item R
\newline Операция циклического сдвига
\item L
\newline Логические операции И и ИЛИ
\item S
\newline Замена на S-блоке
\item X
\newline Применение операции XOR над двумя фрагментами
\item A
\newline Сложение фрагментов длины m как чисел по модулю $2^m$
\end{itemize}

Различные сочетания этих операций дают различные типы схем. Так, криптосистемы SM4 и Skipjack принадлежат к классу XS, AES можно отнести к классу XLS, а схемы криптосистем Salsa20, ChaCha и CAST-128 относятся к классу ARX.

Целью данной работы является исследование так называемых XS схем с целью нахождения оптимальной с точки зрения как криптостойкости, так и быстродействия схемы.

